% !Mode:: "TeX:UTF-8"
% !TEX program  = xelatex
\title{Chapter 4 Summary}
\author[MA320 Group]{
    \begin{tabular}{ll}
        1. & Iydon Liang \\
        2. & Zeyu Dong \\
        3. & Yi'an Yu \\
        4. & Daoyuan Lai \\
        5. & Guoyang Qin
    \end{tabular}
}
\date{\today}
\institute[SUSTech]{
    Department of Mathematics \\
    Southern University of Science and Technology
}

\begin{frame}
	\maketitle
\end{frame}


\begin{frame}[t]{Summary}
	\only<1->{
		\begin{block}{Content of each part}
			This chapter describes the mechanics of the publication process, which includes
			\begin{enumerate}
				\item choosing a journal,
				\item submitting a manuscript,
				\item and the refereeing process.
			\end{enumerate}
		\end{block}}
	\only<2->{
		\begin{block}{Name change}
			\centering
			\begin{tikzcd}[ampersand replacement=\&]
				article \arrow[r, "\substack{\text{after submitting}\\.}"] \&
				manuscript \arrow[r, "\substack{.\\\text{after it has been}\\\text{accepted for publication}}"'] \&
				paper \\
			\end{tikzcd}
		\end{block}}
\end{frame}


\section{Choosing a Journal}
\begin{frame}[t]{Choosing a Journal}
	\only<1->{
		\begin{block}{}
			Choosing an appropriate journal is the most important thing before submission, which requires to consider the following factors:
			\begin{itemize}
				\item the length of the review time,
				\item and the interval from receipt to publication.
			\end{itemize}
		\end{block}}
	\only<2->{
		\begin{block}{Valuable points from the textbook}
			\begin{itemize}
				\item Do not post articles that are not worth publishing.
				\item Do not pursue the reputation of a journal: the value of a paper is not being recognized when it is published, but being continually cited later.
			\end{itemize}
		\end{block}}
\end{frame}


\section{Submitting a Manuscript}
\begin{frame}[t]{Submitting a Manuscript}
	\only<1->{
		\begin{block}{Process of submission}
			Before submitting your manuscript, you should
			\begin{enumerate}
				\item read carefully the instructions for authors that are printed in each issue of your chosen journal;
				\item write a cover letter and deliver it with your manuscript to the chosen journal.
			\end{enumerate}
		\end{block}}
	\only<2->{
		\begin{block}{}
			In addition, most of the requirements are common sense and are similar for each journal, please see page 129 in \emph{Handbook of Writing for the Mathematical Science} for more information.
		\end{block}}
\end{frame}


\section{Refereeing Process}
\begin{frame}[<+->][t]{Refereeing Process}
	\begin{block}{}
		\begin{enumerate}
			\item If a manuscript is submitted to ``The Editor'' of the chosen journal, it is passed on to the editor-in-chief, who assigns it to a member of the editorial board.
			\item Then the chosen editor writes to two or more people asking them to referee the manuscript, suggesting a deadline about six weeks from the time they receive the manuscript.
			\item Finally, when all the referee reports have been received, the editor decides the fate of the manuscript, informs the author, and notifies the journal.
		\end{enumerate}
	\end{block}
\end{frame}

\begin{frame}[<+->][t]{Rejection Letter}
	\begin{block}{}
		If you receive a rejection letter for any reason, you should treat it objectively. At this point, you have three options:
		\begin{enumerate}
			\item When the content of your article does not meet the chosen journal field, consider submitting it to other journals; 
			\item Modify your article according to the review comments and then submit it again;
			\item Write a rebuttal to the editor to prove why the reviewer is wrong.
		\end{enumerate}
	\end{block}
\end{frame}


\section*{Acknowledgments}
\begin{frame}
    \centering\Huge Thank you!
\end{frame}


\section*{References}
\begin{frame}{References}
	\begin{thebibliography}{9}\large
        \bibitem{C:1} Higham, N. J. 1998. Handbook of writing for the mathematical sciences, Society for Industrial and Applied Mathematics, Philadelphia.
        \bibitem{C:2} 汤涛 and 丁玖. 2013. 数学之英文写作, 高等教育出版社, 北京.
    \end{thebibliography}
\end{frame}
