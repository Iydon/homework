% !Mode:: "TeX:UTF-8"
% !TEX program = xelatex
\begin{question}{4.2}{}
    A spherical shell with inner radius 1 and outer radius 2 has a steady-state temperature distribution $u$. Its inner boundary is held at $100$\si{\degreeCelsius}. Its outer boundary satisfies $\pdv{u}{\vb{n}}=-\gamma$, where $\gamma$ is a constant.
    \begin{enumerate}[label=(\alph*)]
        \item Find the temperature $u$.\footnote{Hint: Everything is radial and hence so is $u$ (i.e.\ $u(r, \theta) = u(r)$).}
        \item What are the hottest and coldest temperatures?
        \item Can you choose $\gamma$ so that the temperature on the outer boundary is $20$\si{\degreeCelsius}.
    \end{enumerate}
\end{question}
\begin{enumerate}[label=(\alph*)]
    \item Because of the radial symmetry of the shell and the boundary conditions, the solution $u$ is radially symmetric, i.e.\ $u=u(r)$. Then it follows that
        \[
            u''(r) + \frac{2}{r}u'(r) = 0, \quad 1\leq r\leq 2.
        \]
        The general solution of this ODE is given by $u(r)=a/r+b$. Now the boundary conditions yield $u(1)=a+b=100$ and $\eval{\pdv{u}{\vb{n}}}_{r=2}=u'(2)=-a/4=-\gamma$. Hence, $a=4\gamma$, $b=100-4\gamma$, and as a result,
        \begin{equation}\label{E:4.2-1}
            u = \frac{4\gamma}{r} + 100 - 4\gamma.
        \end{equation}
    \item It follows easily from \eqref{E:4.2-1} that the inner boundary has the hottest temperature which is $100$\si{\degreeCelsius} and the outer boundary has the coldest temperature which is $(100-2\gamma)$\si{\degreeCelsius} if $\gamma\geq 0$; that the outer boundary has the hottest temperature which is $(100-2\gamma)$\si{\degreeCelsius} and the inner boundary has the coldest temperature which is $100$\si{\degreeCelsius} if $\gamma<0$.
    \item Again by \eqref{E:4.2-1} we have $\eval{u}_{r=2}=100-2\gamma$. Thus, once we choose $\gamma=40$ then the temperature on the outer boundary is $20$\si{\degreeCelsius}.
\end{enumerate}


\begin{question}{4.3}{}
    Suppose that $u$ is harmonic function in disk $D=\{r<2\}$ and that $u=3\sin(2\theta)+1$ for $r=2$. Without finding the solution, answer the following questions:
    \begin{enumerate}[label=(\alph*)]
        \item Find the maximum value of $u$ in $\bar{D}$.
        \item Calculate the value of $u$ at the origin.
    \end{enumerate}
\end{question}
\begin{enumerate}[label=(\alph*)]
    \item By maximum principle, we know that the maximum value of $u$ in $\bar{D}$ is attained on the boundary $\partial D$. Since for $r=2$, we have $u=3\sin(2\theta)+1$, which clearly has the maximum value $4$. Thus, the maximum value of $u$ in $\bar{D}$ is $4$.
    \item Using the mean value property for harmonic functions, we obtain that
        \[
            u(0,0) = \frac{1}{4\pi}\int_{\abs{\vb{x}}=2}u(\vb{x})\dd\vb{x} = \frac{1}{4\pi}\int_0^{2\pi}[3\sin(2\theta)+1]\cdot 2\dd\theta = 1.
        \]
\end{enumerate}


\begin{question}{4.4 \index{Unsolved}}{}
    Find the Green's function $G(M;M_0)$ for the Dirichlet problem in the first quadrant of plane:
    \[
        \begin{cases}
            -\Delta G = \delta(M-M_0) & \text{in $\Omega=\{(x,y)\mid x>0, y>0\}$}; \\
            G(0,y) = 0 & \text{for all $y\geq 0$}; \\
            G(x,0) = 0 & \text{for all $x\geq 0$};
        \end{cases}
    \]
\end{question}


\begin{question}{4.5}{}
    Select a suitable method to solve the following boundary value problems:
    \begin{enumerate}[label=(\arabic*)]
        \item \[
                \begin{cases}
                    \Delta u=0 & \text{in $\Omega=\{(r,\theta)\mid 0\leq r<R, 0\leq\theta<2\pi\}$} \\
                    u(R,\theta)=A\cos\theta & \\
                \end{cases}
            \]
        \item \footnote{Hint: the region and B.C.\ are radially symmetric and hence the solution should be radially symmetric.}\[
                \begin{cases}
                    \Delta u=1 & \text{in $\Omega=\{(r,\theta)\mid 0\leq r<R, 0\leq\theta<2\pi\}$} \\
                    u(R,\theta)=0 & \\
                \end{cases}
            \]
        \item \footnote{Hint: For each fixed $r$, $u(r, \theta)$ is $2\pi$−periodic function of $\theta$ which can be expanded by the eigenfunctions with $2\pi$-period B.C.. Thus $u$ takes the form of\[u(r,\theta)=A_0(r)+\sum_{n=1}^\infty(A_n(r)\cos(n\theta)+B_n(r)\sin(n\theta)).\]}\[
                \begin{cases}
                    \Delta u=1 & \text{in $\Omega=\{(r,\theta)\mid 0\leq r<R, 0\leq\theta<2\pi\}$} \\
                    u(R,\theta)=0 & \\
                \end{cases}
            \]
        \item \[
                \begin{cases}
                    \Delta u=0 & \text{in $\Omega=\{(x,y)\mid 0<x<\pi, 0<y<\pi\}$} \\
                    u(0,y)=0 & u(\pi,y)=\cos^2y \\
                    u_y(x,0)=0 & u_y(x,\pi)=0
                \end{cases}
            \]
        \item \[
                \begin{cases}
                    \Delta u=0 & \text{in $\Omega=\{(x,y)\mid 0<x<a, 0<y<b\}$} \\
                    u(0,y)=0 & u(a,y)=0 \\
                    \eval{\left(\pdv{u}{y}+u\right)}_{y=0}=0 & u(x,b)=g(x)
                \end{cases}
            \]
    \end{enumerate}
\end{question}
\begin{enumerate}[label=(\arabic*)]
    \item We know the general solution of Laplace equation in polar coordinates as \eqref{E:4.5-1}.
        \begin{equation}\label{E:4.5-1}
            u(r,\theta) = C_0 + D_0\ln r + \sum_{n=1}^\infty(C_nr^n+D_nr^{-n})(A_n\cos(n\theta)+B_n\sin(n\theta)).
        \end{equation}
        Note that the desired solution $u(r,\theta)$ must be bounded on $\bar{\Omega}$. Therefore, we obtain
        \begin{equation}\label{E:4.5-2}
            u(r,\theta) = C_0 + \sum_{n=1}^\infty r^n(A_n\cos(n\theta)+B_n\sin(n\theta)).
        \end{equation}
        Now the boundary condition implies
        \[
            u(R,\theta) = C_0 + \sum_{n=1}^\infty R^n(A_n\cos(n\theta)+B_n\sin(n\theta)) = A\cos\theta,
        \]
        from which it follows $A_1=A$ and others equal zero. By \eqref{E:4.5-2}, we get the solution
        \[
            u(r,\theta) = \frac{A}{R}r\cos\theta = \frac{A}{R}x.
        \]
    \item Since $\Omega$ and the boundary condition are radially symmetric, we expect that $u$ be radially symmetric: $u=u(r)$. Then we have
        \[
            u''(r) + \frac{1}{r}u'(r) = 1.
        \]
        That is,
        \[
            \left(ru'\right)' = r,
        \]
        from which it follows $ru'=r^2/2+C$ and as a result
        \[
            u(r) = \frac{r^2}{4} + C\ln r + D.
        \]
        Notice that we have $C=0$ since the solution $u$ is bounded. The boundary condition $u(R,\theta)=0=R^2/4+D$ gives us $D=-R^2/4$. Therefore, the solution is given By
        \[
            u = \frac{r^2-R^2}{4}.
        \]
    \item For each fixed $r\in[0,R]$, $u(r,\theta)$ is a periodic function of $\theta$ with period $2\pi$. Thus $u(r,\theta)$ can be expanded by eigenfunctions of the eigenvalue problem
        \[
            \begin{cases}
                X''(\theta) + \lambda X(\theta) = 0 & \theta\in(0, 2\pi), \\
                X \text{ is periodic with period $2\pi$.} & \\
            \end{cases}
        \]
        Therefore,
        \[
            u(r,\theta) = A_0(r) + \sum_{n=1}^\infty (A_n(r)\cos(n\theta) + B_n\sin(n\theta)).
        \]
        Substituting this into
        \[
            u_{rr} + \frac{1}{r}u_r + \frac{1}{r^2}u_{\theta\theta} = \frac{A}{2}r^2\sin(2\theta),
        \]
        we obtain
        \[
            \begin{aligned}
                A_0''(r) &+ \frac{1}{r}A_0'(r) + \sum_{n=1}^\infty\left[\left(A_n''(r) + \frac{1}{r}A_n'(r) - \frac{n^2}{r^2}A_n(r)\right)\cos(n\theta)\right. \\
                &+ \left.\left(B_n''(r) + \frac{1}{r}B_n'(r) - \frac{n^2}{r^2}B_n(r)\right)\sin(n\theta)\right] = \frac{A}{2}r^2\sin(2\theta), \\
            \end{aligned}
        \]
        from which it follows
        \[
            \begin{cases}
                A_0''(r) + \frac{1}{r}A_0'(r) = 0 & 0\leq r<R \\
                B_2''(r) + \frac{1}{r}B_2'(r) - \frac{A}{r^2}B_2(r) = \frac{A}{2}r^2, 0\leq r<R,
            \end{cases}
        \]
        and all the other $A_n$ and $B_n$ are equal to zero. The first equation has general solution
        \[
            A_0(r) = a\ln r + b.
        \]
        To get a particular solution for $B_2$, we use the trivial solution $B_2(r)=Cr^\alpha$, then we have
        \[
            C[\alpha(\alpha-1)+\alpha-4]r^{\alpha-2} = \frac{A}{2}r^2.
        \]
        Therefore, $\alpha=4$ and $C=A/24$. Now the general solution for $B_2$ is
        \[
            B_2(r) = cr^2 + dr^{-2} + \frac{A}{24}r^4.
        \]
        As a result,
        \[
            u(r,\theta) = a\ln r + b + \left(cr^2 + dr^{-2} + \frac{A}{24}r^4\right)\sin(2\theta).
        \]
        Since $u$ is bounded, $a=0=d$. Notice that the boundary condition gives
        \[
            0 = u(R,\theta) = b + \left(cR^2+\frac{A}{24}R^4\right)\sin(2\theta),
        \]
        from which it follows $b=0$ and $c=-AR^2/24$. Finally, we obtain the solution
        \[
            u(r,\theta) = \frac{A}{24}r^2(r^2-R^2)\sin(2\theta).
        \]
    \item We use the method of separation of variables. Assume $u(x,y)=X(x)Y(y)$. Then we have $0=\Delta u=u_{xx}+u_{yy}=X''Y+XY''$, from which it follows
        \begin{equation}\label{E:4.5-3}
            -\frac{X''}{X} = \frac{Y''}{Y} \equiv -\lambda.
        \end{equation}
        Combined with the given boundary condition, we obtain the following eigenvalue problem for $Y$,
        \[
            \begin{cases}
                Y''(y) + \lambda Y(y) = 0, \\
                Y'(0) = Y'(\pi) = 0.
            \end{cases}
        \]
        Thus, $\lambda_n=n^2$ and $Y_n(y)=\cos(ny)$ for $n=0,1,2,\ldots$. Then from \eqref{E:4.5-3}, one must have $X''(x)-n^2X(x)=0$. The boundary condition $u(0,y)=0$ implies that $X(0)=0$. As a result, we get
        \[
            \begin{cases}
                X_0(x) = ax, & \text{if $n=0$,} \\
                X_n(x) = B_n(e^{nx}-e^{-nx}), & \text{if $n=1,2,\ldots$}.
            \end{cases}
        \]
        Now we obtain the solution
        \[
            u(x,y) = ax + \sum_{n=1}^\infty B_n(e^{nx}-e^{-nx})\cos(ny).
        \]
        Plugging in the boundary condition $u(\pi,y)=\cos^2(y)=(1+\cos(2y))/2$ gives us $a=1/(2\pi)$, $B_2=1/\left(2(e^{2\pi}-e^{-2\pi})\right)$, and others equal zero. Finally, we obtain the solution
        \[
            u(x,y) = \frac{1}{2\pi}x + \frac{e^{2x}-e^{-2x}}{2(e^{2\pi}-e^{-2\pi})}\cos(2y).
        \]
    \item Similar to the previous problem, by assuming $u(x,y)=X(x)Y(y)$, we obtain
        \[
            \frac{X''}{X} = -\frac{Y''}{Y} \equiv -\lambda.
        \]
        Together with the boundary condition, we get the following eigenvalue problem
        \[
            \begin{cases}
                X''(x) + \lambda X(x) = 0, \\
                X(0) = X(a) = 0,
            \end{cases}
        \]
        from which it follows
        \[
            \begin{cases}
                \lambda_n = \left(\frac{n\pi}{a}\right)^2 & \\
                X_n(x) = \sin\left(\frac{n\pi}{a}x\right), & n=1,2,\ldots. \\
            \end{cases}
        \]
        Also notice that the ODE for $Y$,
        \[
            Y''(y) - \left(\frac{n\pi}{a}\right)^2Y(y) = 0,
        \]
        has the general solution
        \[
            Y_n(y) = C_n e^{\frac{n\pi}{a}y} + D_ne^{-\frac{n\pi}{a}y},\quad n=1,2,\ldots.
        \]
        Therefore, by superposition principle, we obtain that
        \begin{equation}\label{E:4.5-4}
            u(x,y) = \sum_{n=1}^\infty \sin\left(\frac{n\pi}{a}x\right)\left(C_ne^{\frac{n\pi}{a}y} + D_ne^{\frac{n\pi}{a}y}\right).
        \end{equation}
        Now we have to determine the coefficients $C_n$ and $D_n$. By direct calculations,
        \[
            \eval{\left(\pdv{u}{y}+u\right)}_{y=0} = \sum_{n=1}^\infty \sin\left(\frac{n\pi}{a}x\right)\left[\frac{n\pi}{a}(C_n-D_n)+C_n+D_n\right] = 0.
        \]
        Therefore, we must have
        \begin{equation}\label{E:4.5-5}
            \frac{n\pi}{a}(C_n-D_n) + C_n + D_n = 0.
        \end{equation}
        On the other hand,
        \[
            u(x,b) = \sum_{n=1}^\infty\sin\left(\frac{n\pi}{a}x\right)\left(C_ne^{\frac{n\pi}{a}b} + D_ne^{-\frac{n\pi}{a}b}\right) = g(x),
        \]
        from which we derive
        \[
            \frac{a}{2}\left(C_ne^{\frac{n\pi}{a}b} + D_ne^{-\frac{n\pi}{a}b}\right) = \int_0^a g(x)\sin\left(\frac{n\pi}{a}x\right) \dd x.
        \]
        Combined with \eqref{E:4.5-5}, we see that
        \begin{equation}\label{E:4.5-6}
            \begin{cases}
                C_n = \frac{2(n\pi-a)}{a\left[(n\pi-a)e^{\frac{n\pi}{a}b}+(n\pi+a)e^{-\frac{n\pi}{a}b}\right]} \int_0^a g(x)\sin\left(\frac{n\pi}{a}x\right) \dd x, \\
                D_n = \frac{2(n\pi+a)}{a\left[(n\pi-a)e^{\frac{n\pi}{a}b}+(n\pi+a)e^{-\frac{n\pi}{a}b}\right]} \int_0^a g(x)\sin\left(\frac{n\pi}{a}x\right) \dd x.
            \end{cases}
        \end{equation}
        Thus, the solution of our original problem is given by \eqref{E:4.5-4}, where the coefficients $C_n$ and $D_n$ are determined by \eqref{E:4.5-6}.
\end{enumerate}


\begin{question}{4.6 (Bonus) \index{Unsolved}}{}
    Find the solutions that depend only on $r$ of the \textbf{Helmholtz equation} $-\Delta u=\lambda^2 u$ in 3-D, where $\lambda>0$ is a constant.\footnote{Hint: Define a new function $Z(r)=u(r)/\sqrt{\lambda r}$, then let $r=\rho/\lambda$. The resulting equation has the form of standard Bessel differential equation of order $1/2$, which has analytic solution as a series expansion.}
\end{question}


\begin{question}{4.7}{}
    Show that there is no solution of
    \[
        \begin{cases}
            \Delta u=f, & \text{in $\Omega\subset\RR^3$}, \\
            \pdv{u}{\vb{n}}=g, & \text{on $\partial\Omega$}, \\
        \end{cases}
    \]
    unless
    \[
        \int_\Omega f\dd\vb{x} = \oint_{\partial\Omega} g\dd S.
    \]
\end{question}
Assume that $u$ is a solution of the given BVP. Then by divergence theorem we have
\[
    \int_\Omega f\dd\vb{x} = \int_\Omega\Delta u\dd\vb{x} = \int_\Omega\grad\cdot\grad u\dd\vb{x} = \oint_{\partial\Omega} \pdv{u}{\vb{n}} \dd S = \oint_{\partial\Omega} g \dd S,
\]
which proves the desired result.


\begin{question}{4.8 \index{Unsolved}}{}
    Let $\Omega$ be a bounded domain in $\RR^n$ with smooth boundary. Consider Poisson equation with Neumann boundary condition
    \[
        \begin{cases}
            -\Delta G_N(\vb{x};\vb{x}_0) = \delta(\vb{x}-\vb{x}_0), & x\in\Omega \\
            \pdv{G_N}{\vb{n}}(\vb{x};\vb{x}_0) = \const C, & x\in\partial\Omega,
        \end{cases}
    \]
    where $\vb{n}$ is the unit outer normal of $\partial\Omega$, $\vb{x}_0$ is a fixed point in $\Omega$. DO this problem formally.
    \begin{enumerate}[label=(\roman*)]
        \item Find the value of $\const C$ such that the above BVP has a solution.
        \item By use $G_N$, find a formula for $u(\vb{x}_0)$, where $u$ is a solution of
            \[
                \begin{cases}
                    -\Delta u = f(\vb{x}), & \vb{x}\in\Omega, \\
                    \pdv{u}{\vb{n}} = g(\vb{x}), & \vb{x}\in\partial\Omega.
                \end{cases}
            \]
    \end{enumerate}
\end{question}


\begin{question}{4.9 \index{Unsolved}}{}
    Consider Poisson equation
    \[
        \begin{cases}
            -\Delta u = f(\vb{x}), &\vb{x}\in\RR^3, \\
            \lim_{\abs{\vb{x}}\to\infty} u = 0, & \\
        \end{cases}
    \]
    where
    \[
        f(\vb{x}) = \begin{cases}
            1, & \abs{\vb{x}}\leq 1 \\
            0, & \abs{\vb{x}} > 1. \\
        \end{cases}
    \]
    \begin{enumerate}[label=(\roman*)]
        \item Solve his equation (leave your answer as an integral).
        \item\label{I:4.9-1} Find $\lim_{\abs{\vb{x}}\to\infty} \abs{\vb{x}}u(\vb{x})$.
        \item Let $c$ be the limit found in \ref{I:4.9-1}. Then
            \[
                u(\vb{x}) \approx \frac{c}{\abs{\vb{x}}}
            \]
            for $\abs{\vb{x}}$ large. Interpret this physically.
    \end{enumerate}
\end{question}


\begin{question}{4.10 (Harnack inequality) \index{Unsolved}}{}
    Let $u$ be a nonnegative harmonic function in $\RR^n$. Prove that
    \begin{equation}\label{E:4.10-1}
        \sup_{\RR^n} u \leq 2^n\inf_{\RR^n} u.
    \end{equation}
    \footnote{Hint: take an arbitrary pair of points $P$ and $Q$. Let $R=\abs{P-Q}$}. Use the mean value property of harmonic functions on the balls $B_R(P)$ and $B_{2R}(Q)$ (balls centered at $P$ and $Q$ with radius $R$ and $2R$, respectively).
\end{question}


\begin{question}{4.11 (Liouville Theorem)}{}
    Prove that any harmonic function $u$ in the whole $\RR^n$ that is either bounded from below or above must be a constant function.\footnote{Hint: You may consider wither either $u-\inf u$ or $\sup u-u$.}
\end{question}
Use the Harnack inequality. Assume first that $u$ is bounded below. Applying \eqref{E:4.10-1} to the nonnegative harmonic function $u-\inf u$, we then obtain
\[
    \sup(u-\inf u) \leq 2^n\inf(u-\inf u) = 2^n(\sup u - \sup u) = 0,
\]
which yields that $u\equiv\sup u$.


\begin{question}{4.12 (Decay rate of harmonic functions) \index{Unsolved}}{}
    Suppose $u$ is harmonic in exterior of the ball $B_R(0)$ in $\RR^3$ such that it decays at infinity: $\lim_{\abs{x}\to\infty}u(x)=0$.
    \begin{enumerate}[label=(\roman*)]
        \item Define $v(x)=MG_0(x)-u(x)$, where $G_0$ is the fundamental solution of Laplace equation, and the constant $M$ is taken large enough such that $v>0$ on $\partial B_R(0)$. Prove that $v$ is positive in the exterior of $B_R(0)$.\footnote{Hint: argue by contradiction and use the strong maximum principle.}
        \item Prove that $u$ decays at infinity at least as fast as the fundamental solution.\footnote{Hint: What you need to show is that there is some constant $C>0$ and large $R_0>0$ such that \[\frac{\abs{u(x)}}{G_0(x)}\leq C,\] for $\abs{x}\geq R_0$.}
    \end{enumerate}
\end{question}
