% References
% https://tex.stackexchange.com/questions/199789/which-bold-style-is-recommended-for-matrix-notation

% Packages
\usepackage{hyperref}
    \hypersetup{
        pdfauthor={Iydon Liang},
        pdfcreator={Microsoft® Word 2017},
        pdfproducer={Microsoft® Word 2017},
        colorlinks=true,
        linkcolor=black,
    }
\usepackage{tcolorbox}
    \tcbuselibrary{theorems,breakable}
    \newtcbtheorem{solution}{Solution}{
        colback=white,colframe=orange!90!black,fonttitle=\bfseries,
        arc=2mm,separator sign={\ $\blacktriangleright$},
        breakable}{S}
\usepackage{geometry}
    \geometry{margin=1in}
\usepackage{tcolorbox}
    \tcbuselibrary{breakable,minted}
    \newtcbinputlisting{\pythonfile}[2]{
        listing engine=minted,minted style=trac,minted language=python,
        minted options={numbers=left,breaklines},
        title=\texttt{#2},listing only,breakable,
        left=6mm,right=6mm,top=2mm,bottom=2mm,listing file={#1}}
\usepackage[Glenn]{fncychap}
\usepackage[titletoc]{appendix}
\usepackage{amsmath,amssymb,physics,unicode-math}
\usepackage{nicematrix}
\usepackage{graphicx}
\usepackage{enumitem}

% Symbols
\DeclareMathOperator*{\argmax}{arg\,max}
\newcommand{\RR}{\symbb{R}}
\newcommand{\vx}{\vb{x}}
\newcommand{\vy}{\vb{y}}
\newcommand{\mA}{\vb{A}}
\newcommand{\mX}{\vb{X}}
