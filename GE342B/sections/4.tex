% !Mode:: "TeX:UTF-8"
% !TEX program = xelatex
\section{结论}
综上所述,工具理性是“术”,是方法论,指明了效率最大化的途径;而价值理性是“道”,是核心价值观,为个人的世俗成功提供了道德支撑,为社会的效率最大化提供了精神动力。换句话说,工具理性回答了如何做,怎么做,而价值理性则告诉人为什么要这么做,这么做合不合理,道不道德。人是社会关系的集合,拥有七情六欲,与大环境产生交集的过程中,产生疑惑再正常不过,在快马加鞭奔向成功的终点时,必得有一些坚定不移的个人信念,一套自恰的道德逻辑框架,为成功路上的一切实践提供合理性,避免陷入自我怀疑的怪圈。一个完整的人是一个能够自我认同的人。对社会而言,我们追求的不仅仅是社会效率的最大化,同时还要兼有人文关怀。生产力与生产关系的矛盾决定了,在社会总体生产力不断向前发展的时候,会有无助的弱者被社会轰鸣向前发展的列车甩在后面。如果仅仅依赖工具理性,那么“弱者”是否会因为与前进的生产力相矛盾而被抛弃、被怨恨、呢?如此一来,长期以往积累的阶级矛盾会让社会系统奔溃。恰恰是价值理性的存在,兼顾了发展和人文关怀,我们需要的不仅仅是一个高度发达的工具社会,更是一个在落魄的时候,依然有社会力量为人托底的人文社会。因此工具理性和价值理性不可分割,相辅相成。
