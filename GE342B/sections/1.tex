% !Mode:: "TeX:UTF-8"
% !TEX program = xelatex
\section{文章排版规范及工具声明}
\begin{itemize}
    \item 文章采用排版规范如下:
        \begin{itemize}
            \item 本文主体语言为中文,正文部分排版遵循 \emph{Chinese Copywriting Guidelines for Better Written Communication}\cite{liu2019chinese},文后参考文献著录规则遵循 \gbt\cite{wiki2018文后参考文献著录规则} 标准;
            \item 参考内容如果存在官网,则直接引用官网相应部分,否则一律引用维基百科\cite{wiki2019wikipedia};
            \item 中文字体族采用文泉驿\cite{wiki2019wqy}开源汉字字体项目,英文字体族采用 Garamond\cite{wiki2019garamond} 开源字体项目;
            \item 中文书名采用书名号标明(例如《摩尔根〈古代社会〉一书摘要》),英文书名采用 \emph{Italic Font} 标明(例如上文 \emph{Chinese $\ldots$}),英文术语采用 \texttt{Typewriter Font}(例如上文 \gbt)。
        \end{itemize}

    \item 文章采用工具声明如下:
        \begin{itemize}
            \item 本文排版采用 \LaTeX\cite{latex2019website} 专业学术排版系统;
            \item 涉及到代码采用 \python\ 编程语言,且遵循 \pepeight\cite{rossum2013pep} 标准、\mitlicense\cite{wiki2019mit} 协议。
        \end{itemize}
\end{itemize}

此外,文章的字数统计采用附录~\ref{A:wordcount} 的脚本(即只统计 \unicode{4E00} 至 \unicode{9FA5},以及 \unicode{00B7}、\unicode{00D7}、\unicode{2018}、\unicode{2019}、\unicode{201C}、\unicode{201D}、\unicode{2026}、\unicode{3001}、\unicode{300C}、\unicode{300D}、\unicode{FF01}、\unicode{FF08}、\unicode{FF09}、\unicode{FF0C}、\unicode{FF1A}、\unicode{FF1B}、\unicode{FF1F} 的\textbf{中文字符}及\textbf{中文标点}),因此本文字数统计的结果为 \wordcount。
