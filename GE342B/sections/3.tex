% !Mode:: "TeX:UTF-8"
% !TEX program = xelatex
\section{工具理性在互联网时代的合理性}
因为本章节为个人结合学习实践的经验,在科学技术哲学的指导下进行的思考,所以本章节的引用文献会较前几章明显减少,但这并不意味着本章节缺乏资料的支持。互联网是全世界共同面对的课题,正如许多文章曾探讨过的,互联网是双刃剑,如果利用好可以提高人类的效率进而产生巨大的价值,但是如果利用不好,反而阻碍社会的进步进而带来负面的影响。

首先,我们抛开互联网来看一个例子。前几日转载于朋友圈的一篇文章《武大教授:如何迅速成为著名教授 ?》,里面切实谈到了学术界乃至社会的一些弊端。对于硕士博士来说,
\begin{enumerate}[label=第\chinese*选择:]
    \item \textbf{从政},掌握了话语权,学问跟着级别“长”;
    \item \textbf{经商},掌握了资本,学术跟着银子走;
    \item \textbf{学术},不只专心学术,还要讲究方法。
\end{enumerate}
而最后的学术途径,要想称为著名教授,单纯靠努力是没有用的,如果为了道德等不符合实际的自身价值,则会限制“学术”的发展,文中指明了三招:
\begin{enumerate}[label=〖第\chinese*招〗:]
    \item \textbf{搏命著书,绝不立说。} 拉帮结派,先广后精;
    \item \textbf{糊涂上课,煽情演讲。} 留下好口碑,“因材施教”;
    \item \textbf{出国镀金,进京扬名。} 宣传自己,提高竞争力。
\end{enumerate}

至于文章的细节可以自行搜索,本文的重点是,这套方法乃是典型的工具理性,即对效率的追求。开头一针见血地说出了各个出路的优先级,符号工具理性,有一说一绝不掩饰。这些话有着明确的目标——成为著名教授,同时制定了达到目标最有效最精确的途径,即三招,直面当今学术圈的问题,虽然这些话公开来说有着引起反省的作用,但是本文只考虑在私下里说这些话的理性判断。

同时我们假设如果是价值理性,我们应该作何选择?如果我们的价值观告诉我们要专心学术而不管外界风雨动摇,那么除非个人实力拔群、或者机遇了得,才能在学术圈崭露头角,毕竟学术水平到达一定高度,评价的人就少了,而评价的人互相认识也会阻碍评价的可观与理性。因此如果完全按照自己的价值观,很少能像工具理性那般“容易”,而如果不达到一定水平,是无法改变现实。按照王鼎钧的话来说就是
\begin{quotation}
    咱们的文化,给成功的人架了个框框,做成这个框框的材料就是道德。不管你手上有多少血,或是你口袋里有多少肮脏钱,最后得钻进这个框框,才成正果。钻框框的人得先有个“入围”的资格,就是所谓“成功”。\cite{wiki2019王鼎钧}
\end{quotation}
这话太过文学,如果按照“工具理性”与“价值理性”的话来讲,就是价值理性的实现,必须依赖工具理性。在学术圈里,学术能力必须先有评价机制(如投票规则),才能实现实质上的公平评价。如果连公平评价的形式都不具备,谈何实质?
\begin{quotation}
    总体上说,只要有一种价值理性的存在,就必须存在相应的工具理性来实现这种价值的预设。没有工具理性,价值理性的实现就是空中楼阁。\cite{mba2019工具理性}
\end{quotation}

至于互联网时代,可以说是人性并未发生改变,因此工具理性与价值理性的关系并未发生改变,所以我们不难突出工具理性在互联网时代也是不可抛弃的一部分。但是由于互联网的迭代更新速度较以往存在大幅度提升,所以存在更多的机遇。在互联网时代来临之后,迭代更新速度的大幅度提高导致了工具理性加速资源的不平等现象,因此工具理性更快地变成了支配、控制人的力量,例如现如今的 996 上班制度,这是工具理性极大膨胀的必然结果。为了在互联网时代生存,更应该注重利用工具理性,但是也不可割裂工具理性与价值理性的本质联系,将价值理性与工具理性二者进行统一,不断确证“人是人的最高本质”。
