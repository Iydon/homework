\section{LPPL模型}
\begin{frame}[t]{LPPL模型的具体函数形式}
    该模型拟合的泡沫呈现以下状态:市场价格为对数周期震荡且呈现幂律法则加速,系统越靠近临界点会出现一连串的频率逐渐增加的震荡循环。

    LPPL模型的具体函数形式如下:
    \begin{align*}
        \ln\left[p(t)\right] &= A + B(t_c-t)^m + C(t_c-t)^m\cos\left[\omega\ln(t_c-t)+\varphi\right] \\
                             &= A + B(t_c-t)^m + C_1(t_c-t)^m\cos\left[\omega\ln(t_c-t)\right] \\
                             &\hspace{8em} + C_2(t_c-t)^m\sin\left[\omega\ln(t_c-t)\right]
    \end{align*}
\end{frame}

\begin{frame}[t]{符号解释}
    \begin{table}[H]
    \centering
        \begin{tabular}{@{}ccl@{}}
        \toprule
        符号        & 约束条件            & 解释                                          \\ \midrule
        $p(t)$    & -               & 在$t$时刻的资产价格或资产价格指数;                         \\
        $t_c$     & -               & 临界时间,即泡沫破裂的时间;                              \\
        $t$       & $<t_c$          & 是泡沫破灭前的任一时刻;                                \\
        $A$       & $>0$            & 泡沫持续到临界时间$t_c$时刻$\ln[p(t_c)]$的值;            \\
        $C$       & -               & 围绕指数增长的波动振幅的比例因子;                           \\
        $B$       & $<0$            & \tabincell{l}{为$C$接近0时,$\ln[p(t_c)]$在崩盘时刻$t_c$之\\前的单位时间增长量;} \\
        $m$       & $=0.33\pm 0.18$ & 幂率增长的指数;                                    \\
        $\omega$  & $=6.26\pm 1.56$ & 泡沫期间波动的频率;                                  \\
        $\varphi$ & $\in[0,2\pi]$   & 相位参数。                                       \\ \bottomrule
        \end{tabular}
        \caption{符号说明}\label{T:symbols}
    \end{table}
\end{frame}

\begin{frame}[t]{LPPL模型的假设条件}
    \begin{enumerate}
        \item 投资者是多种多样的,并且不同的投资者的投资策略、对市场信息的判断不一样,同时他们的交易频率和投资头寸也不一样。\\[0.5cm]
        \item 投资者处在一个会互相影响的网络中,投资者的决策会受其他投资者的影响。\\[0.5cm]
        \item 金融崩溃出现在不同的投资者纷纷跟风做同样决策之后。随着市场的发展完善,由于市场反馈机制良好,异质投资者之间会出现自组织的合作,从而导致模仿行为。随着市场协同效应和一致性的加深,泡沫继续扩大。当达到临界时间时,市场上所有的投资者同时采取统一的投资策略。
    \end{enumerate}
\end{frame}
