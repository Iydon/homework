优点:

在问题一中,我们从出租车司机的角度出发,运用线性模型,同时兼顾到了出租车司机的利益与机场客流的疏通能力。基于蒙特卡罗模拟,综合天气等因素,较为完善地对机场乘客的数量变化规律进行了预测。

在问题二中,我们利用了深圳机场及出租车的相关数据,建立了服务布局系统,且对模型的合理性及相关因素的依赖性进行了良好评估,

在问题三中,我们建立多服务台排队模型,对国内现有的服务布局系统进行了优缺点对比。同时利用排队系统的费用决策模型对排队系统进行优化设计,通过绘图方式,清晰、直观地给出了问题的解决方案。

在问题四中,我们给出的方案是具有实用性的,因为我们对国内各大型机场都进行了调研并总结他们处理该类问题的优良办法,从而给出了可直接用于现实环境下的多种方案。

缺点:

研究中将枢纽内出租车上客区的排队系统进行了简化,仅参考了上客区的乘客排队情况。而现实中,出租车排队服务系统通常存在双端排队模式。因此,希望在后续的研究中可以综合考虑多个因素,如出租车排队情况、候客出租车停车位的设置情况等。在此基础上对排队系统进行优化,进而提高出租车乘客的离站效率。


推广:

本模型虽然存在一些不足之处,但是得到的结果还是较为合理的。本模型还可以用于火车站、大型商场等区域。在指标的方面再考虑全面,得到的结果会更让人满意。本模型基于蒙特卡罗模拟,在数据库表现方面良好,之后结合大数据、互联网+等,可解决网约车匹配等多方面问题。
