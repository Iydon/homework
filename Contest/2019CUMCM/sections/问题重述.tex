% !Mode:: "TeX:UTF-8"
% !TEX program  = xelatex
出租车已成为大多数机场乘客选择出行的主要交通工具。国内大多数机场``出发''与``到达''通道分开,送客到机场的出租车司机面临着 1)排队等待接客,2)直接放空返回市区 两种选择。

在某时间段抵达的航班数量和``蓄车池''里已有的车辆数是司机的可观测信息。通常司机可依据个人经验判断某季节或某时间段抵达航班的多少和可能乘客数量的多寡,从而决定是否在机场周边接客。机场出租车管理人员负责``分批定量''放行出租车进入``乘车区'',安排一定数量的乘客与司机双向匹配。而现实中仍有很多影响出租车司机决策的确定和不确定因素。

\begin{enumerate}[label=(\arabic*)]
    \item 分析与出租车司机决策相关因素的影响机理,即各影响因素对司机的期望收益间的数量关系,建立出租车司机选择决策模型,并给出司机的选择策略。
    \item 基于问题一所得模型,收集国内某一机场及其所在城市出租车的相关数据,给出该机场出租车司机的选择方案,并分析模型的合理性和对相关因素的依赖性。
    \item 机场落客与载客区周边常会出现出租车排队和乘客排队的情况。某机场``乘车区''现有两条并行车道,应如何设置最佳``上车点''、合理安排出租车和乘客,以保证车辆和乘客安全且使总乘车效率最高?
    \item 机场的出租车载客收益与载客的行驶里程有关,出租车司机可多次往返载客但不可拒载。管理部门拟对某些短途载客再次返回的出租车给予一定的``优先权'',使得这些出租车的收益尽量均衡(即在司机的时间成本损失上做出一定弥补),试给出一个可行的``优先''安排方案。
\end{enumerate}
