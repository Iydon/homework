% !Mode:: "TeX:UTF-8"
% !TEX program  = xelatex
\begin{abstract}
出租车已成为大多数机场乘客选择出行的主要交通工具。国内大多数机场“出发”与“到达”通道分开,出租车司机面临着 1)排队等待接客,2)直接放空返回市区 两种选择。
%为提高乘客服务效率、减少出租车运力浪费、降低司机等待成本及空载风险,设计出科学的司机载客选择方案与合理的机场周边出租车接送客调度及管制规则变得愈发重要。
本文以深圳宝安国际机场为例,从影响机场乘客与出租车司机作出双向选择的若干因素出发,构建司机决策模型;针对宝安机场外部车道构造,提出合理的乘落客区域规划方案。

针对问题一,本文从影响司机乘客双向选择的因素(乘客与司机角度)出发,运用蒙特卡罗模拟出中国45座主要城市的892条航线,估计出各机场各时间段客流量。运用多元线性回归模型,得出司机留下等待接客的概率可写为:$ P = \gamma_0 + \delta_1t + \delta_2m + \delta_3weather + \beta_1T + \beta_2S_{out} + \eta_t$。计算司机期望收益$E(\tilde{w}) = Pw_1 + (1-P)w_2$,若$E(\tilde{w})>0$,则司机应选择等待载客,且此情况下应进一步优化策略使得$E(\tilde{w})$最大;反之,司机可选择空载返回市区。

针对问题二,本文以深圳宝安国际机场为例,搜集到深圳市出租车计价规则、深圳市2016--2018年天气报告等数据,通过模拟法并验证,得到机场航班准点率及客流量,运用问题一所提出的线性模型(并推广至机器学习算法),训练模型得到最终策略。

针对问题三,本文考虑到多种国内机场的并行两车道分布实际情况(包括:机场内部电梯通道、出发入口、到达出口等各设施的具体分布),结合排队理论,综合考虑到\texttt{M/M/1},纵列\texttt{M/M/S},并列\texttt{M/M/S}三种方案,最终得出符合问题三的高效率设计方案:设置6个可同时搭载乘客的出租车候客服务台时,乘客排队等待时间成本与车道设置成本之和构成的总费用最小。

针对问题四,本文参考上海市现行机场司机多次往返载客方案,本文建议启用智能化匹配管理系统、发放电子短途票、追加日往返机场补贴三个方向,弥补出租车司机等待载客的时间成本损失,保障该类司机的“优先权”,同时建议将所需数据持久化,避免决策无数据支持。

本文编程语言为Python;深圳市天气预测数据来源为深圳市气象局官网;机场准点率信息来源为VariFlight;问题一、二中所需中国主要城市航线数模拟及各时段旅客流量由蒙特卡洛模拟及剪枝得出。


\keywords{蒙特卡洛模拟\quad 机器学习算法\quad 排队论\quad 系统最优原理\quad 边际收益递减}
\end{abstract}
