\subsection{排队模型建立}
根据排队论的相关理论,对枢纽内多点式出租车上客区这一排队系统建立排队模型 。当上客区排队系 统处于全忙期,且系统的服务强度$ρ< 1$时,系统达到稳定状态且不会形成无限排队的现象,在此基础上对系统的输入与输出进行建模。

在系统达到稳态时,$C$个服务台工作,系统中出租车乘客数为$n$的概率如下:
\begin{equation}\label{eq:排队系统出租车乘客数0}
    P_0(C) = \left[\sum_{k=0}^{C-1}\frac{1}{k!}\left(\frac{\lambda}{\mu}\right)^k+ \frac{1}{C!}\frac{1}{(1-\rho)}\left(\frac{\lambda}{\mu}\right)^C\right]^{-1}
\end{equation}

\begin{equation}\label{eq:排队系统出租车乘客数n}
    P_n(C) = \begin{cases}
        \left(\lambda/\mu\right)^kP_0(C)/n, & n=1, 2, \ldots, C \\
        \left(C!C^{n-C}\right)^{-1}\left(\lambda/\mu\right)P_0(C), & n=C+1
    \end{cases}
\end{equation}

用系统中乘客排队的队长$L_s$及其逗留时间$W_s$对系统进行分析,得:
\begin{equation}\label{eq:排队系统乘客队长-2}
    L_s = L_q + C_\rho = \frac{1}{C!}\frac{(C\rho)^C\rho}{(1-\rho)^2}P_0 + \frac{\lambda}{\mu},
\end{equation}

\begin{equation}\label{eq:排队系统逗留时间}
    E(W_s) = \frac{P_n(C)}{C\mu(1-\rho)^2} = \frac{n\mu}{n!(n\mu-\lambda)^2}\left(\frac{\lambda}{\mu}\right)^nP_0(C).
\end{equation}


\subsection{排队系统优化}
利用排队系统的费用决策模型对排队系统进行优化设计。假设乘客等待时间的总费用为$Z_1=\alpha L_s$ ,上车点建设成本为$Z_2=\beta C$,其中$\alpha$为每个乘客单位时间的等待时间成本,$\beta$为单个服务台的服务时间成本与单个上车点的建设费用。当则需要满足两者之和最小才能使得系统进一步优化,即:
\begin{equation}\label{eq:最小化成本}
    \min:Z(C) = Z_1 + Z_2 = \alpha L_s(C) + \beta C, \quad Z(C-1)\leq Z(C)\leq Z(C+1)
\end{equation}