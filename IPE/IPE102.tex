% !Mode:: "TeX:UTF-8"
% !TEX program  = xelatex
\documentclass{ctexart}
\title{马克思主义基本原理概论——马克思的辩证唯物主义认识论}
\author{梁钰栋,11711217}
\date{\today}
\begin{document}
\maketitle

《关于费尔巴哈的提纲》在马克思哲学发展史上具有重要的地位。提纲寥寥千字,系统地论述了科学的实践观。通读全文,我发现乍看之下十一条提纲形式上虽然独立,内核却始终是一致的,“实践的观点”贯穿全文。马克思犀利地指出旧唯物主义的缺陷,即受动性、直观性和不彻底性。与之对应的,马克思指出了主体的主观能动性、抽象性。
马克思主义哲学把实践引入认识论,把辩证法应用于反映论,创立了能动的反映论,科学地揭示了认识的本质,指出认识是在实践的基础上主体对客体的能动的反映。马克思认为,认识是主体对客体的反映,主体对客体的反映受到客观条件的制约,一定客观对象规定一定认识的指向和内容。主体对客体的能动反映是以实践为中介而实现的,认识是在实践的基础上主体对客体能动的反映。同时,主体对客体的反映是一个能动的创造性的过程。

在马克思提出马克思主义新世界观之前,哲学家创立了两条根本对立的认识路线:一条是以费尔巴哈为代表的的“从物到感觉、思想”的唯物主义认识路线,即唯物论的反映论;一条是以黑格尔为代表的“从感觉、思想到物”的唯心主义认识路线,即唯心主义的先验论。

辩证唯物主义认识论是强调主观能动的反映论,而旧唯物主义认识论是片面的机械反映论,二者虽然有共同点,但毫无疑问是对立的。辩证唯物主义的认识论和旧唯物主义的认识论都坚持从物到感觉和思想的认识路线,认为认识是主体对客体的反映。马克思的《关于费尔巴哈的提纲》则深刻地指出了二者的不同之处。

第一,辩证唯物主义的认识论把科学的实践观引进认识论,认为认识是在实践的基础上主体对客体能动的反映。旧唯物主义的认识论否认实践在认识中的作用,把认识看作是主体对客体消极、被动的反映。也就是《提纲》中第一条指出的,“从前的一切唯物主义包括费尔巴哈的唯物主义的主要缺点是:对事物、现实、感性,只是从客体的或者直观的形式去理解,而不是把它们当作感性的人的活动,当作实践去理解,不是从主体方面去理解。”

其次,辩证唯物主义的认识论把辩证法引入认识论,科学地说明了认识发展的辩证过程。而旧唯物主义认识论不懂认识的辩证法,否定认识的辩证过程,认为认识是一次完成的。

第三,辩证唯物主义的认识论是能动的革命的反映论。旧唯物主义的认识论是消极、被动、直观的反映论。马克思认为,世俗基础使自己从本身中分离,并在云霄中固定为一个独立王国,这只能用这个世俗基础的自我分裂和自我矛盾来说明。因此,对于世俗基础本身首先应当从它的矛盾中去理解,并在实践中使之革命化。因此,例如,自从发现了神圣家族的秘密在于世俗家庭之后,世俗家庭本身就应当在理论上和实践中被消灭。
\end{document}
