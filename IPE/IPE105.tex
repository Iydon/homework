% !Mode:: "TeX:UTF-8"
% !TEX program  = xelatex
\documentclass{ctexart}
\title{形势与政策——一国两制}
\author{梁钰栋,11711217}
\date{\today}
\begin{document}
\maketitle

\section{什么是“一国两制”}
一国两制,即“一个国家,两种制度”,是我国政府为实现国家和平统一而提出的基本国策。按照邓小平的论述,“一国两制”是指在一个国家的前提下,国家的主体坚持社会主义制度,香港、澳门、台湾保持原有的资本主义制度长期不变。“一国两制”是邓小平为实现祖国统一而创造的方针,是中华人民共和国政府在台湾问题上的主要方针,也是香港、澳门两个特别行政区所采用的制度。香港问题、澳门问题和台湾问题都是历史上遗留下来的问题,解决这些问题,实现国家统一,是中华民族的共同愿望。

\section{当下香港的乱局}
香港持续发生的激进暴力犯罪行为,严重践踏法治和社会秩序,严重破坏香港繁荣稳定,严重挑战“一国两制”原则底线。越来越多的香港市民已经看得很清楚,围绕修改《逃犯条例》所出现的事态已经完全变质。少数暴徒用他们的违法犯罪行为向世人表明,他们的目的、他们的矛头所向,已与修例无关。他们心甘情愿充当外部势力和反中乱港势力的马前卒,不惜做出暴力违法的恶行,目的就在于搞乱香港、瘫痪特区政府,进而夺取特区的管治权,从而把香港变成一个独立或半独立的政治实体,假高度自治、“港人治港”之名行完全自治、对抗中央之实,最终使“一国两制”名存实亡。止暴制乱、恢复秩序是香港当前最紧迫的任务。中国政府维护国家主权、安全、发展利益的决心坚定不移,贯彻“一国两制”方针的决心坚定不移,反对任何外部势力干涉香港事务的决心坚定不移。

\section{某些港独势力如何曲解“一国两制”}
一些人在谈论“两制”的时候不是在谈社会主义制度和资本主义制度,而是在谈香港的自治权,谈香港与中央的分权关系。在一些人的理解中,“两制”就意味着中国有着两种不同的中央与地方关系的权力安排,两种不同的国家结构形式。这是严重误解,也是香港回归以来极端反对派能够鼓动或裹挟普通市民上街游行的一个重要原因。反对修订《逃犯条例》事件中的暴徒和鼓吹者不是在维护香港的资本主义制度,而是在要谋求从中央“分权”,实现香港的某种独立性。实际上,“两制”这个概念已经被歪曲为香港应当实行跟内地不同的国家结构形式制度,香港应当像联邦制下的政治实体一样拥有固有的权力。这是一个非常严重的认识误区,而这种认识误区经常被人挑动起来对抗中央政府。须知香港与内地城市的差别仅在于中央授予了香港更多的自治权,但无论香港的自治权有多大,都是来自于中央的授予。只有破除香港社会中对“一国两制”的错误幻像,只有澄清了“一国两制“的真实含义,香港才能实现长治久安。

\section{“一国两制”的实践典范:澳门}
澳门回归祖国二十年来,在中央政府的大力支持下,在广大澳门居民的共同努力下,澳门的各项事业取得了全面进步,取得的成就有目共睹。“一国两制”和基本法,既能保证澳门特别行政区有“一国之本”,有强大的国家做后盾和支撑;又能保证澳门特别行政区有“两制之利”,保持高度自治和地域优势,从而实现长期繁荣稳定。澳门成功实践“一国两制”的最宝贵经验,就是全面准确实施宪法和基本法,其核心是澳门对国家政治体系、国家核心价值的准确理解和真心拥护,并始终保持和国家的紧密互动与融合,充分发挥“一国两制”的独特优势以及中央政府的政策支持,从而实现了澳门的稳定和繁荣。
\end{document}
