% !Mode:: "TeX:UTF-8"
% !TEX program  = xelatex
\documentclass{ctexart}
\usepackage{url}
\title{毛泽东思想和中国特色社会主义理论体系概论}
\author{梁钰栋,11711217}
\date{\today}
\begin{document}
\maketitle

\section{《中国社会各阶级的分析》读后感}
《中国社会各阶级的分析》中,作者首先开宗明义,提出最首要的问题:“谁是我们的敌人,谁是我们的朋友”。为何有此一问呢?当时的背景是党内存在两种倾向,一种是以陈独秀为代表,只注重同国民党合作,忘记了农民,这是右倾机会主义;另一种是以张国焘为代表,只注重工人运动,同样忘记了农民。这两种机会主义都感觉自己力量不足,而不知去何处寻找力量,到何处去取得广大的同盟军。毛泽东是是农民的儿子,深深了解农民在种种压迫下建立了一种天然的反抗精神。他认为有结合小农佃户及雇工,以反抗牵制中国的帝国主义者,打倒军阀及贪官污吏,反抗地痞劣绅,以保护农民之利益,而促进国民革命之必要。

为了让党内早点认识到“中国无产阶级最广大和最忠实的同盟军是农民”,认识到无产阶级是革命的领导力量,解决中国革命的领导、动力、对象等一系列问题,毛泽东决定写一篇文章来厘清这一系列问题。毛泽东以其超凡的洞察力,通过分析当时中国的国民情况,他一针见血地指出:中国无产阶级最广大和最忠实的同盟军是农民,这样就解决了中国革命中最主要的同盟军问题。

这篇文章从“阶级”角度出发,分析各阶级的政治态度,这种分析方法是开创性的。同时作者并未把重心放在各阶级生产关系的生产关系属性上,而是突出内外民族矛盾、贫富阶级矛盾,从动态的角度考察各阶级的政治态度以划清敌我界限,为党的发展指明了道路。

在一一分析各个阶级对革命的态度后,毛泽东做了一个总体概括,“一切勾结帝国主义的军阀、官僚、买办阶级、大地主阶级以及附属于他们的一部分反动知识界,是我们的敌人。工业无产阶级是我们革命的领导力量。一切半无产阶级、小资产阶级,是我们最接近的朋友。那动摇不定的中产阶级,其右翼可能是我们的敌人,其左翼可能是我们的朋友———但我们要时常提防他们,不要让他们扰乱了我们的阵线。”

令人印象深刻的是,毛泽东对各阶级的分析不仅来自于他对各阶级的洞察,更有坚实的实践基础。毛泽东回老家韶山养病期间,广泛接触群众,开夜校教农民识字、学珠算、进行马克思主义思想启蒙,并且在农民革命觉悟提高后成功地发动了农民运动。正是这样的成功实践让毛泽东的《分析》有了强大的理论和实践基础,指出了一条真正适合中国国情的革命道路。在新时代下,这种分析方法依然具有极大的指导作用。


\section[由贸易战引发的对文化自信的思考]{由贸易战引发的对文化自信的思考——学习四个意识、四个自信和两个维护有感}
\subsection{四个意识,四个自信和两个维护的具体含义}
“四个意识”是指政治意识、大局意识、核心意识、看齐意识,这“四个意识”是2016年1月29日中共中央政治局会议最早提出来的。习近平总书记在庆祝中国共产党成立95周年大会上的讲话强调,全党同志要增强政治意识、大局意识、核心意识、看齐意识,切实做到对党忠诚、为党分忧、为党担责、为党尽责。

“四个自信”即中国特色社会主义道路自信、理论自信、制度自信和文化自信,由习近平总书记在庆祝中国共产党成立95周年大会上提出,是对党的十八大提出的中国特色社会主义“三个自信”的创造性拓展和完善。

“两个维护”是指坚决维护以习近平总书记党中央的核心、全党的核心地位,坚决维护党中央权威和集中统一领导。

\subsection{由贸易战引发的对文化自信的思考}
中国的发展是不可阻挡的历史潮流,然而现任美国政府却奉行“美国优先”政策,无视中美经济结构、发展阶段特点和国际产业分工的现实,坚持认为中国采取不公平、不对等的贸易政策,导致美国出现对华贸易逆差,使之在双边经贸交往中“吃了亏”\footnote{中华人民共和国国务院新闻办公室:《关于中美经贸磋商的中方立场》,2019 年 06 月 02 日。},并对华采取加征关税措施,悍然挑起贸易战。2019 年4 月 29 日美国国务院政策规划主任斯金纳表示:“与苏联的竞争,在某种程度上是西方家庭内部的争斗。这是我们第一次面临一个非白人的强大竞争对手。”“是与一种完全不同的文明和不同意识形态之间的竞争,美国以前从未经历过。”\footnote{童黎:《美国:我们第一次面对一个非白人的强大竞争对手》,2019 年 05 月 06 日。}——尽管斯金纳本人也是黑人。然而历史终将证明,美国想要消灭世界政治文化的多样性,让全世界摒弃本国的实际情况而采纳同一种政治体系是绝对行不通的,事实上人类文明的多样性就是其延续不绝的力量。正如新加坡总理李显龙所说:“大家必须接受中国会继续壮大的事实,并且了解组织中国继续壮大是不可能的事,更非明智之举。”\footnote{环球网,\url{https://world.huanqiu.com/article/9CaKrnKkMLt},2019.06.01。}美国人何以对中国崛起如此敏感,他们对中国崛起的反应有多少是来自于冷静理智的分析,又有多少是来自于对非白种文化文明的严重不适呢?美国一向号称是世界上最包容最开放的国家,何以接受不了一种与之不同的文明以和平发展的方式崛起呢?我们可能永远也不会知道真正的答案,因为理智与情感之间的这些严峻斗争,很可能正在潜意识深处展开。

另一方面,尽管美国如此迫不及待地阻止中国经济的发展,然而在世界经济全球化如此深入的今天,美国真的能做到和中国脱钩吗?答案是否定的。从美国前驻华公使傅立民这段话中可以一窥奥秘:“美国每年有 65 万从事科学和工程专业的学生毕业,其中很大一部分是中国人。若排挤他们,美国将损失大量科技人才,预计到 2025 年,中国所拥有的熟练技术工人的数量,将超过经合组织所有成员国的总和。若与中国脱钩,意味着美国疏远的是世界上科学家、技术专家、工程师和数学家数量最多的国家。若切断中美科技交流,与其说会阻碍中国的技术进步,不如说将会损害美国自身的创新能力。”更何况,中国是美国的第一大贸易伙伴,中美贸易占美国贸易总额的16\%以上,美国每年消费的商品和原材料有很大一部分来自中国,并且这些商品或原材料只有中国生产,或者中国这些商品和原材料占据世界市场绝大部分,美国有足够的准备与中国脱钩吗?当然不行。实际上,美国目前表现出来的讨价还价、出尔反尔不过是一种谈判手段,目的是为了试探中国在贸易战的问题上能有多少让步,以期最大化地在贸易战中获利,扭转美国对中国的贸易逆差。美国从未想过与中国真正脱钩,借机敲打中国、抹黑与他们不同的意识形态才是其目的所在。

事实上某些西方媒体热衷于歪曲事实以抹黑中国,由来已久。比如2019 年英国货车惨案中,事件还未经调查,CNN 便迫不及待宣布遇害者“全部为中国人”。我们的第一反应是为 39 条人命惋惜悲哀,某些西方媒体却如同打了兴奋剂,认为是抓住了可以籍以抨击中国制度的机会。他们难以公正的、客观的角度看待中国如今的发展,在中国发展越来越好的当下,依旧以不思改变的姿态,保持对中国的偏见和恶意。

然而事情正在发生变化。李子柒的视频在海外社交媒体上已经坐拥七百多万粉丝,传统之美和时代烙印在不自觉间被她融入视频中,她的视频实际是把中国文化中的“山水田园梦”具象化了。如今的李子柒俨然成了中国文化外宣的“旗舰店”,无数国外网友被视频背后的中国文化所深深打动。

诚然,文化的影响力是国家影响力的重要体现。一方面,中华民族拥有五千年绵延不绝的辉煌历史文化,这在世界历史上都是独一无二的。汉朝的铁骑风沙、大唐的万国来朝;唐诗之豪迈,宋词之婉约;南国的杏花烟雨,北国的千里冰封……华夏文明早已深深地融入了我们的血肉中。另一方面,身处西方文化席卷的潮流,对我们中国青年而言,保持对中华文化的历史认知、现实了解和情感认同则更为重要。我们新的一代出生在国家经济实力日益强大的年代,感受到的是欣欣向荣繁荣发展的中国,我们对近代中国的落后和屈辱有深刻了解但并不被历史包袱绑架,因此对中华文明的认可是从小就镌刻在骨子里的。

中国特色社会主义先进文化已深深融入国民的日常生活中。我们坚持文化自信,就是要激发党和人民对中华优秀传统文化的历史自豪感,在全社会形成对社会主义核心价值观的普遍共识和价值认同。这一点从当今电影市场的格局便可窥一斑。美国的漫威英雄大片大行其道,日本的《名侦探柯南》好评如潮,这却丝毫不妨碍《战狼》塑造出的孤胆英雄深入人心,《哪吒之魔童降世》掀起国风动漫的狂潮。这便是当今中国青年的生命力、包容力。

于我们而言,文化差异并不可怕,既不是让一个国家故步自封、自高自大、拒绝与其他文化形态相互交流的借口,更不是借以挑起国与国的非理性经济较量的工具。面临当今世界愈加频繁的文化碰撞,泱泱中华立足于秉承着取其精华为我所用的态度,有足够的底气不敌视另一种文化,也有足够的大国气度包容其他的文化,更有足够的自信坚持中国特色社会主义文化。

\end{document}
